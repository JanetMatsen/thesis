This thesis highlights experimental and computational approaches to understanding cellular metabolism.
Chapter \ref{chapter:A} demonstrates application of transcriptomics to a pure Type II methanotroph culture growing under methane-grown conditions.
Strong conclusions about which metabolic pathways are active were possible given the high-quality reference genome available for RNA-seq alignments.

Chapter \ref{chapter:B} extends these techniques to species rich microbial communities, and uses time-series and experimental differences to probe the driving factors for community composition.
In this case, the only reference DNA available for alignments was a handful of isolate genomes, and contigs assembled from the pooled 88 metagenomes.
The ability to derive high-quality inferences about the genetic potential and gene expression levels of these samples is much lower than for the pure culture case of Chapter \ref{chapter:A}.
A high-level understanding of which microbes dominate the sample is provided, and steps to refine understanding are outlined.

The potential of extracting additional information from the gene-expression levels found in Chapter \ref{chapter:B} is explored in Chapter \ref{chapter:C}.
The ability to use canonical correlation analysis to identify correlations in the gene-expression data was tested on a preliminary data set.
Methods of extimating the gene-expression partial correlation matrix were tested, and aims to identify cross-species gene pairs that co-occur across samples were proposed.
%Metrics for validating such partial correlation networks are proposed, and
%Specific aims for mining these networks for between-species interactions are proposed. % TODO: add more specific aims

In total, this work demonstrates use of wet-lab and computational approaches to improve understanding of microbial metabolism.
The skills I gained are directly useful for guiding metabolic engineering of these strains, or other bacteria.
Furthermore, the data science skills and ability apply machine learning to the unique challenges of biology have broad applications beyond studies of metabolism.
I am thrilled to continue pushing the boundaries of sustainable chemical production using my combination of expertise in microbiology, wet-lab experimentation, and data science.

