\chapter{Introduction}

I came to the Lidstrom Lab eager to learn the art of engineering microbes, with understanding that the future of sustainable chemical production will feature such processes.
During my PhD, I had the opportunity to do significant amounts of challenging wet-lab biology, much of which is not included in the scope of this dissertation.
While working in strain design, engineering, enzyme design, enzyme evolution, and fermentation, it became clear to me that our ability to achieve the potential of our field relies on robotics, high-throughput screening, and advanced analytics.
With Mary's incredible support, I redirected my focus to computational biology and data science.
Along the way, I completed data science coursework and was granted the first Advanced Data Science Option from the University of Washington.
Again with Mary's generous support, I was able to do two computational internships at two leading biotech startups in our field.
Moving forward as a biology-focused data scientist, I am so grateful for the understanding of microbial physiology and biological experiments that the Lidstrom Lab provided me.

This thesis focuses on genomic and transcriptomic studies of pure cultures and mixed microbial populations, as both an experimentalist (Chapter \ref{chapter:A}), and fully computational data scientist (Chapters \ref{chapter:B}, \ref{chapter:C}).
For information about the wet-lab work regarding the Formolase Pathway \cite{siegel2015}, development of SIP3-4 as a model organism, and development of a high-throughput formaldehyde-production enzyme screen, please see my general exam (\url{https://github.com/JanetMatsen/thesis/blob/master/documents/2015_Matsen_General_Exam.pdf}).

\section{Methylotrophy and Methanotrophy}
The focus of the Lidstrom lab is understanding and manipulating methylotrophs, which are microorganisms that grow using reduced single-carbon compounds such as methane, methanol, methylamine and formate as their sole carbon source and energy source \cite{anthony1982,mila2009}.
Methanotrophs are methylotrophs that have extra metabolic modules enabling them to enzymatically convert methane to methanol \cite{kalyuzhnaya2015puri}.
The energetics of activating these highly reduced single carbon compounds and the unique challenge of assimilating the activated products requires special metabolic pathways not present in other microbes.
Furthermore, understanding of both methylotrophs and methanotrophs is increasingly important given their role in mitigating these greenhouse gases, and their biotechnological potential for converting inexpensive single-carbon compounds to economically valuable multi-carbon compounds.

\section{Tools for Metabolism Studies}
The Lidstrom Lab uses many approaches to observe and alter microbial metabolic networks, allowing deeper understanding of the complex network and the potential to perturb it for human good.
Metabolism can be observed at many levels; at each level there are distinct sets of techniques. % to quantify molecular concentrations.
The most commonly quantified entities are DNA pools, RNA pools, metabolite concentrations, and carbon flux.
DNA is used to identify metabolic potential of a single organism or population, in terms of what genes are available.
RNA is used to identify which of these genes microbes actually express, and under which conditions.
Metabolite concentrations and flux measurements indicate the effects of gene expression on the chemical environment of the cell.

Each type of study corresponds to a different scientific discipline of ''omics", the collective characterization and quantification of pools of biological molecules.
The study of genomes is called "genomics", while the study of transcripts (RNA) is called "transcriptomics".
While the Lidstrom Lab combines all of these observational tools and a variety of pathway-altering tools, this thesis focuses on analysis of genomic and transcriptomic datasets obtained via high-throughput sequencing.
%There will be some mention of a metabolite study ("metabolomics") in Chapter Chapter \ref{chapter:A} that complements the discussed transcriptomic study.

%High-throughput sequencing (HTS)

Genomics and transcriptomics can further be classified by whether they address a single organism's nucleic acids or those from a collection of organisms.
When genomics and transcriptomics are used to study populations rather than single organism (pure) cultures, the prefix "meta" is added, yielding "metagenomics" and "metatranscriptomics".
Metagenomics can imply different types of sequencing, which can be coarsely divided into deeper sequencing of only 16S ribosomal DNA to profile taxa abundance, or more broad sequencing of the entire DNA pool in a community.
This dissertation will focus entirely on the latter, termed "shotgun" metagenomics to clarify that specific types of DNA sequences such as 16S or other marker genes are not targeted for sequencing.
This un-targeted sampling allows discovery of the genes available to the population of organisms, and the potential to identify genomes of un-cultured organisms.

\section{Machine Learning}
'Omics studies usually stop after tabulating and describing the measurements made.
For metagenomics and transcriptomics, this corresponds to descriptions of the genetic material present and the expression level of the predicted genes.
There can, however, be much richer descriptions of datasets if explored with the appropriate computational techniques.
Chapter \ref{chapter:C} takes that extra step, and applies machine learning techniques to extract experimentally testable hypotheses from complex meta-'omics data.
%The aim is to generate experimentally testable hypotheses for future studies.

%Road map for rest of thesis:
%\section{Unifying Themes}
%The broad theme of this dissertation is use of high-throughput sequencing to understand microbial metabolism.
%It begins with the simplest and most traditional type of study, using a lab isolate, and transitions to the increased biological and analytical complexity of natural sediment communities.
%For both systems, it aims to identify which metabolic pathways organism use, given species-level and community-level metabolic redundancy.
%For the case of complex communities, additional computational and machine learning techniques are used to extract higher-order function from the communities.

\section{Road Map for Thesis} %: Unifying themes
The broad theme of this dissertation is use of high-throughput sequencing to understand microbial metabolism.
It begins with the simplest and most traditional type of study, using a wild-type lab isolate, and transitions to the increased biological and analytical complexity of natural sediment communities.
For both systems, it aims to identify which metabolic pathways organism(s) use, given species-level and community-level metabolic redundancy.
For the case of complex communities, additional computational and machine learning techniques are used to extract higher-order function from the communities.

Chapter \ref{chapter:A} describes how carbon flows through the metabolic network of a model methanotroph.
This thesis focuses on the the transcriptomics half \cite{matsenOB3b} of a two-part study including metabolomics \cite{yangOB3b} of the same culture.
Pure cultures allow high-confidence elucidation of metabolic pathways, but may not represent the often un-cultivatable wild relatives.

Interest in how methane is oxidized in nature led to Chapter \ref{chapter:B}, which explores metabolism in species-rich methanotroph-enriched natural communities.
It aims to clarify how diverse populations of methanotrophs and auxiliary microbes oxidize methane in nature: %, by analysis of 88 pairs of metagenomic/metatranscriptomic datasets:
Which microbes are important in natural communities?
Are the lab isolates (e.g. Chapter \ref{chapter:A}) good representatives of the wild variants?
Which metabolic pathways do individual microbes use?
What metabolic interactions occur between species in nature?
Can the presence of one organism be linked to a shift in metabolic strategy of another?

This work is part of a rapidly developing field of community `omics.
The newness of the field and corresponding lack of standards, combined with a plethora of tools to navigate requires thoughtful method selection, custom definitions of efficacy, and consideration of trade-offs between seemingly redundant tools.
For this thesis, special attention is paid to monitoring how well the results reflect the ground-truth raw sequences representing each sample.

%Results preview

Chapter \ref{chapter:B} approximates %rarefaction of the natural community into a smaller number of higher-abundance microbes over the 14 week experiment.
the question of "who is there", and highlights the importance of oxygen availability, a key environmental variable, in determining community composition, and community gene expression.
% the dominant methanotrophs and partner microbes under high and low oxygen concentrations, as well as differences in the genes expressed for each condition.
%Insight into community rarefication and abundance stochasticity is illustrated by having four biological replicates at each condition, and 11 samples in each series.
%Differences in how wild strains metabolized compared to their potentially domesticated lab-isolate relatives are revealed.

Chapter \ref{chapter:C} builds on the abundance and gene-expression results of Chapter \ref{chapter:B} by applying machine learning to identify patterns not evident to the human eye.
It highlights portions of the machine learning literature that can be leveraged for analysis of communities, given that the data in this field has a much larger number of features and smaller number of samples than is common in typical machine learning applications.
These methods can provide testable hypotheses about between-species interactions to enhance understanding of community metabolism.

% opportunities to extract meaningful biology from datasets
%It highlights the potential to extract meaningful biology from complex meta-omics data.
%This chapter explores the potential to apply machine learning techniques, which typically involve less noisy data and orders of magnitude more samples.
%It is the hope that the critical eye used to tune methods in Chapter \ref{chapter:B} yield sufficiently clean data for algorithms to identify the underlying structure of the data.




