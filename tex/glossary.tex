\chapter*{Glossary}      % starred form omits the `chapter x'
\addcontentsline{toc}{chapter}{Glossary}
\thispagestyle{plain}
%
\begin{glossary}
\item[argument] replacement text which customizes a \LaTeX\ macro for
each particular usage.
\item[document class] a file of macros that tailors \LaTeX\ for
a particular document.  The macros described by this thesis
constitute a document class.
\item[document option] a macro or file of macros
that further modifies \LaTeX\ for
a particular document.  The option {\tt[chapternotes]}
constitutes a document option.
\item[formatter] (as opposed to a word-processor) arranges printed
material according to instructions embedded in the text.
A word-processor, on the other hand, is normally controlled
by keyboard strokes that move text about on a display.
\item[pica] an archaic unit of length.  One pica is twelve points and
six picas is about an inch.
\item[point] a unit of length.  72.27 points equals one inch.
\item[roman]  a conventional printing typestyle using serifs.
the decorations on the ends of letter strokes.
This thesis is set in roman type.
\item[rule] a straight printed line; e.g., \hrulefill.
 
\end{glossary}

