\chapter*{Glossary} % \chapter*{Glossary} <-- the star turns off the chapter numbering and removes it from the index.
\renewcommand*{\arraystretch}{1.4} % space between entries % http://tex.stackexchange.com/questions/135593/adding-row-spacing-to-a-longtable

\begin{singlespace}
\begin{longtable}{ p{.20\textwidth}  p{.80\textwidth} }

\textbf{AMI} & Amazon machine image, a master image for the creation of virtual servers (known as EC2 instances)
                in the Amazon Web Services (AWS) environment. \\
\textbf{ANI} & Average nucleotide idenity, a measure of similarity between organisms. \\
\textbf{assembly} &  The process of, or result of inferring the original genome fragments that produced sequencing reads. \\
\textbf{AWS} & Amazon Web Services, the leading cloud computing platform. \\
\textbf{AWS instance} & A single computer rented from Amazon Web Services, with user-selected performance characteristics. \\
\textbf{bin} & A collection of contigs that approximates one organism, or a group of closely related strains. \\
\textbf{binning} &  The process of clustering contigs into bins.  \\ % "clustering"?
\textbf{BWA} & Burrows-Wheeler Aligner, a tool for maping reads to reference DNA sequences. \\
\textbf{CCA} & Cannonical correlation analysis. \\
\textbf{CheckM} &  A computational tool used to assess genome bin completeness and contamination. \\
\textbf{contig} & A contiguous DNA sequence.  In this case, they are the longer DNA stretches identified from assembling Illumina reads. \\
\textbf{C1} & Single-carbon compounds such as methanol, \ce{CO_2}, or formate. \\
\textbf{coverage} & number of reads that include a given nucleotide in the reconstructed sequence. \\
\textbf{cross-validation} &  A model validation technique that loops over subsets of the data, and leads to an assessment of how the results of a model will generalize to an independent data set.  \\
\textbf{d} & The dimensionality of the data.  In this thesis, it is usually the number of genes. \\
\textbf{EC2} & A virtual server (computer) in Amazon's Elastic Compute Cloud. \\
\textbf{EFS} & Amazon's Elastic File System, a scalable file storage service.  \\
\textbf{Elviz} & A Joint Genome Analysis toolchain for metagenomics and metatranscriptomics.  \\
\textbf{FASTA} &  A biology sequence data format, which can contain DNA, RNA, or protein sequences for different units of information such as genes, contigs, or whole genomes.  \\
\textbf{fastq} & A \texttt{.fasta}-like file with associated sequence quality information.  This is the most raw output from the sequencing platform handled in this thesis. \\
\textbf{feature} & A single measurable property of a phenomenon being observed, such as one gene's expression level. \\
\textbf{GC} & DNA bases.  When used in terms like "high GC", it indicates high fraction of G, C nucleotides over A, T nucleotides. \\
\textbf{genomics} & The collective characterization and quantification of pools of DNA molecules \\
\textbf{genome bin} & A collection of contigs used to approximate a single organism's genome.  Often referred to as "bin". \\
%\textbf{graph} &  \\
%\textbf{held-out set} &  \\
\textbf{hyperparameter} & Parameters in a machine learning model that express higher-level properties of the model such as the model complexity or the learning rate. \\
\textbf{i.i.d.} & Independent and identically distributed.
    Each random variable has the same probability distribution as the others and all are mutually independent \\
\textbf{isolate} & A single microbial strain that has been isolated and propagated in pure culture. \\
\textbf{JGI} & The Joint Genome Institute, a U.S. Department of Energy laboratory which provided sequencing
                services for the metagenomics work of this thesis \\
\textbf{kb} & Kilobase. A unit of DNA (or RNA) length, corresponding to 1000 base pairs. \\
\textbf{mapping} & Alignment of sequencing reads to target DNA such as genome or contig sequences.\\
\textbf{machine learning} & Models that have the ability to learn without being explicitly programmed. \\
\textbf{MetaBAT} &  A binning tool. \\
\textbf{metagenome} & DNA recovered directly from a mixed microbial community, rather than an isolated strain. \\
\textbf{metatranscriptome} & RNA recovered directly from a mixed microbial community, rather than an isolated strain. \\
\textbf{methanotroph} & A microbe which can use methane as its sole carbon and energy source. \\
\textbf{methylotroph} & A microbe which can use reduced C1 compounds such as methanol as its sole carbon and energy source. \\
\textbf{non-methanotrophic methylotroph} & While all methanotrophs are methylotrophs, not all methylotrophs can utilize methane.
	Such methylotrophs are often described as non-methanotrophic methylotrophs. \\
\textbf{MyCC} & A binning tool. \\
\textbf{N} & The number of samples available for machine learning, e.g. 88 for 88 metatranscriptomes. \\
%\textbf{N50} & A weighted median statistic for assemblies.  %http://biology.stackexchange.com/questions/34122/an-example-for-n50-why-do-we-need-it
%    Defined such that 50\% of the entire assembly is contained in contigs or scaffolds equal to or larger than this value. \\
    %the length for which the collection of all contigs of that length, or longer, contains at least half of the total of the
    %            lengths of the contigs in the assembly” \\ % https://www.ncbi.nlm.nih.gov/pmc/articles/PMC4426941/pdf/bbi-9-2015-075.pdf (cites 49, 50)
\textbf{OB3b} & \textit{Methylosinus trichosporium} {OB3b}, a RuMP cycle methanotroph, that is the subject of study in Chapter \ref{chapter:A}  \\
\textbf{omics} & Refers to fields of study ending in \textit{-omics}, such as genomics and transcriptomics, which quantify pools of biological molecules. \\
\textbf{partial correlation} & A measure of the strength and direction of a linear relationship between two continuous variables whilst controlling for the effect of one or more other continuous variables. \\
\textbf{PCA} & Principle components analysis. \\
\textbf{PhyloPhlAn} & A computational tool that can assign taxonomy to metagenome bins. \\
\textbf{Prokka} & A computational tool for annotating genes on contigs. \\
\textbf{read} & a single short DNA sequence of a larger high-throughput sequencing data set. \\
\textbf{regularization} & A technique to limit model complexity and prevent over-fitting to training data. \\
\textbf{RPKM} & Reads per kilobase per kilobase mapped \cite{mortazavi2008} \\
\textbf{taxa} & A taxanomic category, such as a genus or species. \\
\textbf{taxonomy} & A scheme of classification for organisms. \\
\textbf{tetranucleotide frequency} & The frequency of 4-nucleotide sequences in DNA, which is conserved for an organism. \\
\textbf{TPM} & Transcripts per million \cite{wagner2012}, an alternative to RPKM \cite{mortazavi2008} \\
\textbf{training set} &  Data used to train a machine learning model. \\
\textbf{transcriptomics} & The collective characterization and quantification of pools of RNA molecules (transcripts). \\
\textbf{16S} & A ribosomal protein, whose DNA sequence is commonly used for evolutionary inference. \\

\end{longtable}
\end{singlespace}
