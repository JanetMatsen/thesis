Engineered microbes will play a key role in the transition from fossil fuel derived chemicals to sustainable chemicals.
Successful metabolic reenginnering requires deep understanding of microbial physiology, bioinformatics, and data science.
This thesis utilizes all three to study the metabolism of methane and methanol-utilizing microbes.
Both pure cultures (Chapter \ref{chapter:A}) and complex natural (Chapter \ref{chapter:B}) communities are studied.
The potential to leverage statistical and machine learning for large meta-omics datasets is also explored (Chapter \ref{chapter:C}).
Overall, we are able to make strong conclusions when high-quality isolate genomes are available, however, these inferences are much more difficult in the case for complex microbial communities with unknown underlying genomic composition.
% many opportunities for applying data science to these large data sets, however the strength of our conclusions are limited by the sample size and noise

%Cite \cite{matsenOB3b}, \cite{yangOB3b}
