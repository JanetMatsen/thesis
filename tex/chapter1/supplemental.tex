\section{Supplemental}

Figure S1. RNA-Seq reads mapped per base relative to start of pmo-operon. Left, the average number of reads mapped to the region −330 to −270 upstream from the start of pmoC for two biological replicates. The sequence at each base is shown above the bars. Right, the range (shown in red) and mean of two biological replicates (shown as black line) for the number of reads mapped per base for the coding and downstream region of the pmo-operon.

Figure S2. Genetic organization and relative expression (RPKM) of the mxa gene cluster (A), the pqq gene cluster (B) and cluster of genes encoding reactions of the H4MPT-linked C1 transfer pathway (C).

Figure S3. RNA-Seq reads mapped per base relative to start of mxaF ORF. Left, the average number of reads mapped to the region −200 to −140 upstream from the start of mxaF for two biological replicates. The sequence at each base is shown above the bars. Right, the range (shown in red) and mean of two biological replicates (shown as black line) for the number of reads mapped per base for the coding and downstream region of the mxaF cluster.

Figure S4. RNA-Seq reads mapped per base relative to start of serine cycle gene operon. The log10 average number of reads mapped at each base from biological replicates one and two is shown. In (A), the upstream location spanning −250 to −190 from putative start of ftfL. (B) The expression over the entire operon. Note the several drop to near zero upstream indicating the operon is not co-transcribed. (C) The −230 to −170 region upstream from sga. Bases from the +strand are shown across the top of the figure. (D) The −270 to −210 region upstream of mclA.

Figure S5. RNA-Seq reads mapped per base relative to start of fae1-1. Left, the average number of reads mapped to the region −235 to −185 upstream from the start of fae1-1 gene for two biological replicates. The sequence at each base is shown above the bars. Right, the range (shown in red) and mean of two biological replicates (shown as black line) for the number of reads mapped per base for the coding and downstream region of the fae1-1 and fae1-2 genes.Supplementary Figure S6Primary structure alignment for phosphoenolpyruvate carboxylases (Ppc1 and Ppc2) from M. trichosporium OB3b and Ppc-homologs.

Table S1. Transcripts detected by de novo assembly RNA-Seq data.

Table S2. Gene expression profile in methane-grown cells of M. trichosporium OB3b. Values represent reads per kilobase of coding sequence per million (reads) mapped (RPKM).

Table S3. Summary of putative transcription site mapping (Note that the published sequences need references added).

Table S4. Summary of RNA-seq (Illumina) reads.

Table S5. Genes removed from reference scaffold before alignment.



